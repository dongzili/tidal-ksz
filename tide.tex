While a cosmic signal in 21cm measurement is of the order of mK,  
foregrounds coming from Galactic emissions, telescope noise, 
extragalactic radio sources and Radio recombination lines, 
can reach the order of Kelvin \cite{DiMatteo04}\cite{Masui13}. 
Lots of techniques have been developed to substract the foregrounds, 
taking advantage of the attribute that they have fewer bright spectral
degrees of freedom\cite{Switzer15}.
Unfortunately, the substraction usually contaminates the smooth large scale structure information.
Since the large scale information is essential for the estimate of
peculiar velocity, we need to recover the information.
%Since foregounds are spectrally smooth, when we are substracting them, 
%we lose the information about smooth large scale structure of real signals. 
However, since large scale structure is correlated closely to the emergence of peculiar velocity, we need a method to estimate its distribution.

The cosmic tidal reconstruction is a kind of quadratic statistics developed to achieve this goal.
Its main idea is using small scale filamentary structures to solve for the large scale tidal shear and gravitational potential.

Here, we present a 3 dimentional reconstruction algorithm that works best in close linear regions.

First, we filter for the small scale structures that are most likely to be influenced by tidal force of large scale fields.

(1) Convolve the field with a Gaussian kernal
$S(\bm{k})=e^{-k^2R^2/2}$, 
we take $R=1.25\ \mr{Mpc}/h$ \cite{2012:pen},
to reduce the complicated non-linear effects on small scales.

(2) Gaussianize the field, taking 
$\delta_g=\mathrm{ln}(1+\delta)$. 
This is to allieviate the problem that filter $W_i$ in (3) heavily weights high density regions.

(3) Apply filter $W_i(\bm{k})$, which assigns weights to $\delta_g$ according to predicted displacements caused by tidal field in near linear regions \cite{2015:zhu}.
${\delta}^{w_i}_g(\bm{k})=W_i(\bm{k})\delta_g(\bm{k})$, 
\begin{eqnarray}
\label{eq:wi}
W_i(\bm{k})=i (\frac{P(k)f(k)}{P_{tot}^2(k)})^{\frac{1}{2}}\frac{k_i}{k}
=S(k)\frac{k_i}{k}
\end{eqnarray}
i indicates $\hat x,\hat y,\hat z$ directions, 
$P_{tot}=P+P_{noise}$ is observed matter powerspectrum, 
P is theoretical matter powerspectrum,
$f=2\alpha(\tau)-\beta(\tau)dlnP/dlnk$, 
where $\alpha$ and $\beta$ are functions related to linear growth funcion, and are calculated to be (0.6, 1.3) for $z=1$ and (0.4, 0.9) for $z=2$.
\footnote{The effect of the filter $W_i$ on different scales could be seen in Appendix 1.}


Second, we estimate the tidal force and hense reconstruct the large scale density field.

(1) Following gravitational lensing procedures, we decompose the $3\times3$ symmetric, traceless tidal force tensor into 5 $\gamma$ components, and estimate them from density variance.
\begin{eqnarray}
\label{eq:gamma}
\hat{\gamma}_1(\bm{x})&=&
[{\delta}^{w_1}_g(\bm{x}){\delta}^{w_1}_g(\bm{x})-
{\delta}^{w_2}_g(\bm{x}){\delta}^{w_2}_g(\bm{x})],\nonumber\\
\hat{\gamma}_2(\bm{x})&=&
[2{\delta}^{w_1}_g(\bm{x}){\delta}^{w_2}_g(\bm{x})],\nonumber\\
\hat{\gamma}_x(\bm{x})&=&
[2{\delta}^{w_1}_g(\bm{x}){\delta}^{w_3}_g(\bm{x})],\\
\hat{\gamma}_y(\bm{x})&=&
[2{\delta}^{w_2}_g(\bm{x}){\delta}^{w_3}_g(\bm{x})],\nonumber\\
\hat{\gamma}_z(\bm{x})&=&
[(2{\delta}^{w_3}_g(\bm{x}){\delta}^{w_3}_g(\bm{x})-
{\delta}^{w_1}_g(\bm{x}){\delta}^{w_1}_g(\bm{x})-
{\delta}^{w_2}_g(\bm{x}){\delta}^{w_2}_g(\bm{x}))]/3,\nonumber
\end{eqnarray}
(2) Reconstruct 3D density field.
\begin{eqnarray}
\kappa_\mr{3D}(\bm{k})&=&\frac{1}{k_1^2+k_2^2+k_3^2}
[(k_1^2-k_2^2)\gamma_1(\bm{k})+2k_1k_2\gamma_2(\bm{k})\\
&&+2k_1k_3\gamma_x(\bm{k})+2k_2k_3\gamma_y(\bm{k})\nonumber
+(2k_3^2-k_1^2-k_1^2)\gamma_z(\bm{k})].
\end{eqnarray}

Third, we correct bias and suppress noise with a Wiener filter.
Due to the foregrounds, the noise in z direction will be different from x,y direction, therefore we apply an anisotropic Wiener filter.
\begin{eqnarray}
	\label{eq:wiener}
\hat \kappa_{c}(\bm{k})=\frac{\kappa_{3D}(\bm{k})}{b(k_\perp,k_\parallel)}W(k_\perp,k_\parallel)\ ,
\end{eqnarray}
Bias $b=\frac{P_{k3D \delta}}{P_\delta}$, Wiener filter $W=\frac{P_\delta}{P_{k3D}/b^2}$.

Here and afterwards, 
we use "$\wedge$" to denote recontructed fields.

