While a cosmic signal in 21cm measurement is of the order of mK,  
foregrounds coming from Galactic emissions, telescope noise, 
extragalactic radio sources and Radio recombination lines, 
can reach the order of Kelvin \cite{DiMatteo04}\cite{Masui13}. 
Lots of techniques have been developed to substract the foregrounds, 
taking advantage of the attribute that they have fewer bright spectral
degrees of freedom\cite{Switzer15}.
Unfortunately, the substraction usually contaminates the smooth large scale structure information.
Since the large scale information is essential for the estimate of
peculiar velocity, we need to recover the information.
%Since foregounds are spectrally smooth, when we are substracting them, 
%we lose the information about smooth large scale structure of real signals. 
%However, since large scale structure is correlated closely to the emergence of peculiar velocity, we need a method to estimate its distribution.

The cosmic tidal reconstruction is a kind of quadratic statistics developed to achieve this goal.
Its main idea is using small scale filamentary structures to solve for the large scale tidal shear and gravitational potential.

Detailed steps are as follows.

First, we filter for the part of information corresponding to linear small scale structure from the 21cm density field.

(1) Convolve the field with a Gaussian kernal
$S(\bm{k})=e^{-k^2R^2/2}$, 
we take $R=1.25\ \mr{Mpc}/h$ \cite{2012:pen}\cite{2015:zhu},
and expect it to filter out the nonlinear structures in small scales.

(2) Gaussianize the smoothed field, by taking 
$\delta_g=\mathrm{ln}(1+\delta)$, 
this is to allieviate the problem that the filter we apply in next step heavily weights the high density region.

(3) Convolve with filter $W_i(\bm{k})$ which filters for the small scale structures, 
$W_i(\bm{k})=(\frac{P(k)f(k)}{P_{tot}^2(k)})^{\frac{1}{2}}\hat{k}_i$,
where i corresponds to $k_x,k_y$ directions, $P_{tot}=P+P_{noise}$ is the observed powerspectrum, f is a function related to the redshift and linear growth funcion of the universe, see 
\cite{2015:zhu}.
\footnote{The effect of the filter $W_i$ on different scales could be seen in Appendix 1.}

Second, we reconstruct the large scale density field from tidal shear field.

(1) Estimate the tidal shear fields from density variance.
To avoid error caused by peculiar velocity, we only consider the shear field in tangental plane
(perpendicular to the line of sight).
$\gamma_1=(\delta_g^{w1})^2-(\delta_g^{w2})^2$,
$\gamma_2=2\delta_g^{w1}\delta_g^{w2}$.

(2) Reconstruct 3D density field from Possion equations.
\begin{eqnarray}
\label{eq:k3d}
\kappa_{3D}(\bm{k})=\frac{2k^2}{3(k_1^2-k_2^2)^2}[(k_1^2-k_2^2)\gamma_1(\bm{k}+2k_1k_2\gamma_2(\bm{k})]\ .
\end{eqnarray}

Third, we correct bias and control noise with a Wiener filter.

From Eq.(\ref{eq:k3d}) we can infer that the error of $\kappa_{3D}(\bm{x})$ is 
$\sigma_{k3d}(\bm{k})\propto(\frac{k^2}{k_\perp^2})^2$, where $k_\perp$ refers to modes that are tangental.
The anisotropy in $k_\perp$ and $k_\parallel$ is due to the discard of information about radial shear field.
To control the large noise corresponds to small $k_perp$, we apply filter to calculate the clean large scale density field$\hat \kappa_c$:
\begin{eqnarray}
	\label{eq:wiener}
\hat \kappa_{c}(\bm{k})=\frac{\kappa_{3D}(\bm{k}}{b(k_\perp,k_\parallel)}W(k_\perp,k_\parallel)\ ,
\end{eqnarray}
where the bias $b=\frac{P_{k3D \delta}}{P_\delta}$, Wiener filter $W=\frac{P_\delta}{P_{k3D}/b^2}$, here and afterwards, 
we use $\wedge$ to denote recontructed fields.

